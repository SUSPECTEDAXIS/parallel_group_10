\documentclass[conference]{IEEEtran}
\IEEEoverridecommandlockouts
% The preceding line is only needed to identify funding in the first footnote. If that is unneeded, please comment it out.
\usepackage{cite}
\usepackage{amsmath,amssymb,amsfonts}
\usepackage{algorithmic}
\usepackage{graphicx}
\usepackage{textcomp}
\usepackage{xcolor}
\def\BibTeX{{\rm B\kern-.05em{\sc i\kern-.025em b}\kern-.08em
		T\kern-.1667em\lower.7ex\hbox{E}\kern-.125emX}}
\begin{document}
	
	\title{COP4520 Rough Draft\\
		{}
		\thanks{}
	}
	
	\author{\IEEEauthorblockN{Carlos Sanchez Ruiz}
		
		\and
		\IEEEauthorblockN{Yosha Riley}
		
		\and
		\IEEEauthorblockN{Kariel Sanchez Ruiz}
		\and
		\IEEEauthorblockN{Andrew Ballen}
	}
	
	\maketitle
	
	\begin{abstract}
		For our project we will be tackling the Dining philosophers problem but do it in a language that will be new to us, rust. The rules to solve this is to not let two philosophers have two forks simultaneously, avoid a deadlock and make sure they do not starve, meaning they all must eat. This problem will allow us to not only learn a new language but put into practice concepts from the class. We will be using locks, and threads to be able to solve this problem and have variables that all threads use. For our approach we will have to create an algorithm that can determine the state of a philosopher which are either to think or eat. We will also have to know whenever a philosopher picks up a fork and make sure they do not grab both at once but one at a time. Then finally also have the algorithm allow the philosopher eat once they acquire 2 forks. Once we are done with this project we should be able to learn rust and be able to comfortably code threads in the language to solve complex problems.
	\end{abstract}
	
	\begin{IEEEkeywords}
		component, formatting, style, styling, insert
	\end{IEEEkeywords}
	
	\section{Introduction}
	 The goal of our project is to use concepts we learned in class which involve concurrent objects, mutual exclusion and learn the language RUST. RUST is a coding language ( a systems programming language) that was designed to be safe, fast and allow for concurrency.  It has a user-friendly compiler that with it brings integrated package managers and also prevents crashes in the code. To put it in simpler terms it wont allow the code to run if it sees that it will cause a crash instead it will show in the compiler to the user what changes to make to prevent a crash and allow the code to run. 
	 \linebreak
	 \underline{Advantages of using RUST}
	 \begin{enumerate}
	 	\item Allows for easy and quick debugging as once again it is a fast language.
	 	\item When compared to other languages it will support complex code better with faster runtime.
	 	\item It is safer than C++ and Java.
	 \end{enumerate}
	 \underline{Disadvantages of using RUST}
	 \begin{enumerate}
	 	\item Rust is more complicated than other languages to learn
	 	\item Due to the learning curve it may take longer to solve a problem in RUST
	 	\item There isn’t much research on it in case we need help since it is new
	 	\item There can be memory leaks and RUST and this will affect our runtime to be slower
	\end{enumerate}
	 Overall Rust is good at solving concurrency problems which is why we chose to do the Dining philosophers problem for our project as it is a complex concurrency problem. The Dining philosophers problem states that there are a N amount of philosophers/threads around a table that have an infinite appetite and will never be fuel. They all have a fork to their left and to their right and must have both picked up to be able to eat. As stated before they must not be able to pick up both forks at once, they can either think or eat, and they must not starve which is represented by reaching a deadlock.
	 \linebreak
	 \underline{Common issues of the problem}
	 \begin{enumerate}
	 	\item Two philosophers pick up the same fork
	 	\item One philosopher picks up a fork and another one picks up the only other available fork meaning the philosopher starves (causes a deadlock).
	 \end{enumerate}
	 To solve these common issues we will have to use locks to make sure once a thread uses a fork another one cannot which in theory should solve issue number 1. To solve issue number two we will have to design our code so that the other fork available to a philosopher once they pick one up is inaccessible. This will require for the variable storing forks to be accessible by all threads.
	 \linebreak
	 \underline{How using RUST positively affects this problem}
	 \begin{enumerate}
	 	\item We can share data between the threads using Arc and mutex
	 	\begin{enumerate}
	 		
	 		\item This will let us tell a philosopher if a fork is available for them or not
	 	\end{enumerate}
	 	\item We can spawn child threads in RUST which can make on thread relate to another
	 	\item RUST can lock data meaning that once a thread/philosopher picks up a fork we can lock it until the philosopher drops the fork which then will make it accessible to other threads.

	\end{enumerate}
	 
	
	
\end{document}
